\chapter{绪论}%大标题



\zhlipsum[1]

\section{表格测试}

\begin{table}[h!]
\centering 
\caption{表格示例}
\begin{tblr}{lccr}
	\hline
	Alpha   & Beta  & Gamma  & Delta \\
	\hline
	Epsilon & Zeta  & Eta    & Theta \\
	\hline
	Iota    & Kappa & Lambda & Mu    \\
	\hline
\end{tblr}
\end{table} 

注意,在表格的正文部分,其中的标题应:中文需用黑体「\verb+\heiti+」,英文需要加粗「\verb+\textbf{text}+」





\section{参考文献测试}

张三\cite{叶普1993关于对瞬心的动量矩定理}提出。
张三\cite{爱因斯坦文集2009}提出。
张三\cite{叶普1993关于对瞬心的动量矩定理}提出。
张三\cite{叶普1993关于对瞬心的动量矩定理,2003张量分析,lanczos1986variational}提出。
海上生明月\footnote{脚注测试1},
海上生明月\footnote{脚注测试2,脚注测试2}。
文献\citestyle{numbers}\cite{2003张量分析}中说明。
文献\citestyle{numbers}\cite{叶普1993关于对瞬心的动量矩定理,2003张量分析,lanczos1986variational}中说明。
张三\citestyle{super}\cite{爱因斯坦文集2009,叶普1993关于对瞬心的动量矩定理,2003张量分析,lanczos1986variational}提出。



\section{脚注测试}%二级标题

\zhlipsum[2]
海上生明月\footnote{脚注测试1}
海上生明月\footnote{脚注测试2,脚注测试2}
海上生明月\footnote{脚注测试1}
海上生明月\footnote{脚注测试2,脚注测试2}
海上生明月\footnote{脚注测试1}
海上生明月\footnote{脚注测试2,脚注测试2}
海上生明月\footnote{脚注测试1}
海上生明月\footnote{脚注测试2,脚注测试2}
海上生明月\footnote{脚注测试1}
海上生明月\footnote{脚注测试2,脚注测试2}
海上生明月\footnote{脚注测试1}
海上生明月\footnote{脚注测试2,脚注测试2}
海上生明月\footnote{脚注测试1}
海上生明月\footnote{脚注测试2,脚注测试2}
海上生明月\footnote{脚注测试1}
海上生明月\footnote{脚注测试2,脚注测试2}
海上生明月\footnote{脚注测试1}
海上生明月\footnote{脚注测试2,脚注测试2}
海上生明月\footnote{脚注测试1}
海上生明月\footnote{脚注测试2,脚注测试2}
海上生明月\footnote{脚注测试1}
海上生明月\footnote{脚注测试2,脚注测试2}
海上生明月\footnote{脚注测试1}
海上生明月\footnote{脚注测试2,脚注测试2}

\subsection{理论预言}%三级标题



\zhlipsum[3]

\subsubsection{实验装置}%四级标题

\zhlipsum[4]


\chapter{实验装置原理}%大标题


\section{图片测试}

海上生明月,天涯共此时 good moning。
\zhlipsum[1]


\begin{figure}[h!]
	\centering
	\begin{subfigure}{0.4\linewidth}
		\centering
		\includegraphics[scale=.5]{example-image-a}
		\caption{说明}\label{}
	\end{subfigure}
	\hfil
	\begin{subfigure}{0.4\linewidth}
		\centering
		\includegraphics[scale=.5]{example-image-b}
		\caption{演示}\label{}
	\end{subfigure}
	\caption{图片排版示例}
\end{figure}

\section{公式测试}

劳仑衣普桑,认至将指点效则机,最你更枝。想极整月正进好志次回总般,段
然取向使张规军证回,世市总李率英茄持伴。用阶千样响领交出
\begin{equation}\label{key}
	\Delta x \geqslant \frac{p c}{E} \frac{\hbar}{m c}=\frac{v}{c} \frac{\hbar}{m c}
\end{equation}
最你更枝。想极整月正进好志次回总般,段
然取向使张规军证回,世市总李率英茄持伴。用阶千样响领交出
\begin{equation}\label{key}
	S=\int \mathcal{L}\left(\phi, \partial_{\mu} \phi\right) \mathrm{d}^{4} x
\end{equation}
最你更枝。想极整月正进好志次回总般,段
然取向使张规军证回,
\begin{equation}\label{key}
	\left[\frac{\partial \mathcal{L}}{\partial \phi}\left(\partial_{\nu} \phi\right)+\frac{\partial \mathcal{L}}{\partial\left(\partial_{\mu} \phi\right)} \partial_{\nu}\left(\partial_{\mu} \phi\right)\right] \delta x^{\nu}+\mathcal{L} \partial_{\nu}\left(\delta x^{\nu}\right)=\left(\partial_{\nu} \mathcal{L}\right) \delta x^{\nu}+\mathcal{L} \partial_{\nu}\left(\delta x^{\nu}\right)
\end{equation}
世市总李率英茄持伴。用阶千样响领交出
\begin{equation}\label{key}
	\left(\begin{array}{l}
		\phi_{1} \\
		\phi_{2}
	\end{array}\right) \longrightarrow\left(\begin{array}{l}
		\phi_{1}^{\prime} \\
		\phi_{2}^{\prime}
	\end{array}\right)=\left(\begin{array}{cc}
		\cos \theta & -\sin \theta \\
		\sin \theta & \cos \theta
	\end{array}\right)\left(\begin{array}{l}
		\phi_{1} \\
		\phi_{2}
	\end{array}\right)
\end{equation}
知易众美布会亲军千,件声坑志支较学。农六斯南何记子机量
各然,快写线信权间越部色,象照屈型部物治地长。难要技第对老共达质标压心,
\begin{equation}\label{key}
	\begin{aligned}
		\left\langle q^{\prime}\left|\mathrm{e}^{-\mathrm{i} H\left(t^{\prime}-t\right)}\right| q\right\rangle=& \int\left[\frac{\mathrm{d} p \mathrm{~d} q}{2 \pi}\right] \exp \left\{\mathrm{i} \int_{t}^{t^{\prime}} \mathrm{d} t[p \dot{q}-H(p, q)]\right\} \\
		& \equiv \lim _{n \rightarrow \infty} \int\left(\frac{\mathrm{d} p_{1}}{2 \pi}\right) \cdots\left(\frac{\mathrm{d} p_{n}}{2 \pi}\right) \int \mathrm{d} q_{1} \cdots \mathrm{d} q_{n-1} \\
		& \exp \left\{\mathrm{i} \sum_{i=1}^{n} \delta t\left[p_{i}\left(\frac{q_{i}-q_{i-1}}{\delta t}\right)-H\left(p_{i}, \frac{q_{i}+q_{i-1}}{2}\right)\right]\right\}
	\end{aligned}
\end{equation}
\zhlipsum[8]


\section{理论计算}%二级标题

\zhlipsum[2]

\subsection{理论预言}%三级标题

\zhlipsum[3]

\subsubsection{结果近似}%四级标题

\zhlipsum[4]